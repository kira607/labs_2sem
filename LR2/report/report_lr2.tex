% !TeX root = report_lr2.tex
\documentclass[12pt,a4paper]{article}  % шаблон для статьи, шрифт 12 пт

\usepackage[utf8]{inputenc}  % использование кодировки Юникод UTF-8

\usepackage[compact]{titlesec}  % для titlespacing
% \titlespacing{\заголовок}{слева}{перед}{после}[справа]
\titlespacing*{\section}{0.75cm}{1em}{0.1em}  % отступ заголовка 
\titlespacing*{\subsection}{0.75cm}{1em}{0.1em}

\usepackage{indentfirst}  % отступ первого абзаца
\setlength{\parindent}{0.75cm}

\usepackage[labelsep=endash]{caption}  % тире вместо двоеточия в картинках
\usepackage[russian]{babel}  % пакет поддержки русского языка
\usepackage{graphicx}  % кртинки
\usepackage{color}  % цветной код

\usepackage[T1]{fontenc}
\usepackage{xcolor}  % ???

\usepackage{listings}  % листинги кода из файлов
\lstset{
	language=c++,                   % выбор языка для подсветки (здесь это С++)
	basicstyle=\small,              % размер и начертание шрифта для подсветки кода
	numbers=left,                   % где поставить нумерацию строк (слева\справа)
	numberstyle=\small,             % размер шрифта для номеров строк
	stepnumber=1,                   % размер шага между двумя номерами строк
	numbersep=5pt,                  % как далеко отстоят номера строк от подсвечиваемого кода
	backgroundcolor=\color{white},  % цвет фона подсветки - используем \usepackage{color}
	showspaces=false,               % показывать или нет пробелы специальными отступами
	showstringspaces=false,         % показывать или нет пробелы в строках
	showtabs=false,                 % показывать или нет табуляцию в строках
	frame=false,                    % рисовать рамку вокруг кода
	tabsize=2,                      % размер табуляции по умолчанию равен 2 пробелам
	captionpos=t,                   % позиция заголовка вверху [t] или внизу [b] 
	breaklines=true,                % автоматически переносить строки (да\нет)
	breakatwhitespace=false,        % переносить строки только если есть пробел
	escapeinside={\%*}{*)}          % если нужно добавить комментарии в коде
}

%перенос строк внутри таблиц
\newcommand{\specialcell}[2][c]{%
	\begin{tabular}[#1]{@{}c@{}}#2\end{tabular}}

% нумерация рисунков по номеру главы (e.g. 1.1)
% для отчёта по курсовой
% \renewcommand\thefigure{\arabic{section}.\arabic{figure}}

% Шаблон создания картинки
% \begin{figure}[hpt!]
% 	\centering
% 	\includegraphics[width=0.6\linewidth]{photo/photo1}
% 	\caption{Подпись к рисунку}
% 	\label{имя ссылки на рисунок}
% \end{figure}

\begin{document}
	
	\thispagestyle{empty}
	
	\begin{center}
		\Large{
			\textbf{МИНОБРНАУКИ РОССИИ}
			
			\textbf{Санкт-Петербургский государственный}
			
			\textbf{электротехнический университет «ЛЭТИ»}
			
			\textbf{им. В.И. Ульянова (Ленина)}
			
			\textbf{Кафедра САПР}
		}
	\end{center}
	
	\topskip=0pt
	\vspace*{\fill}
	\begin{center}
		\Large{
			\textbf{
				ЛАБОРАТОРНАЯ РАБОТА №2\\
				по дисциплине «Программирование»\\
				Тема: Функции, обеспечивающие
				обработку массива\\
			}
		}
	\end{center}
	\vspace*{\fill}
	
	\begin{tabular}{lcr}
		Студенты гр. 9892 & \begin{tabular}{p{60mm}} \\ \hline \end{tabular} & Лескин К.А.  \\\\
		                 & \begin{tabular}{p{60mm}} \\ \hline \end{tabular} & Миллер В.В. 
		\\\\
		Преподаватель    & \begin{tabular}{p{60mm}} \\ \hline \end{tabular} & Кузьмин С.А. 
		\\\\
	\end{tabular} 
	
	\begin{center}
		Санкт-Петербург\\
		2020
	\end{center}
	%////////////////////////////////////////////////////////////////////////////////////////////////
	%////////////////////////////////////////////////////////////////////////////////////////////////
	%////////////////////////////////////////////////////////////////////////////////////////////////
	\newpage
	
	\section*{Цель работы}
	
	Получение практических навыков разработки алгоритмов для обработки
	массивов.
	
	\section*{Формулировка задания}
	
	Разработать программу, обрабатывающую элементы многомерного
	массива в соответствии с вариантом.
	Размерности массива должны вводиться пользователем с файла, или же
	с клавиатуры.
	На их основе массивы должны быть созданы в динамической
	памяти программы с помощью указателя.
	Доступ к элементам массива также осуществляется с помощью указетеля.
	Элементы массива могут вводиться пользователем как с файла, так и с
	клавиатуры.
	Исходный и преобразованный массивы должны быть выведены на экран
	и записаны в выходной файл после обработки.
	
	Индивидуальный вариант: Разработать программу для сложения двух 
	прямоугольных матриц (двумерных массивов).
	
	\newpage
	\section*{Форматы входных и выходных файлов}
	
	\subsection*{Входной файл}
	
	Входной файл должен быть представлен в виде текстового файла, содержащего последовательность чисел.
	Разширение файла может быть любым.
	Последовательность должна быть записана в следующем виде:
	Первым членом последовательности должна быть ширина матрицы $ w $ -- количество столбцов,
	далее должна идти высота матрицы $ h $ -- количество строк матрицы.
	После должны идти элементы матрицы.
	Их количество определяется по формуле $ w*h $.
	Между элементами могут быть любые разделительные символы
	(пробел, знак переноса строки, знак табуляции).
	Рекоментуется оформлять элементы
	матрицы в виде сетки $ w*h $ для повышения читаемости файла.
	
	\subsection*{Выходной файл}
	
	Выходной файл записывается в следующем формате:
	
	\begin{itemize}
		\item Матрица 1
		\item Знак '+'
		\item Матрица 2
		\item Знак '='
		\item Матрица 3 (сумма матриц 1 и 2)
	\end{itemize}
	
	\section*{Описание структур данных}
	
	Матрица в программе представлена в виде структуры.
	Структура состоит из:
	
	\begin{itemize}
		\item Ширина матрицы (по умолчанию 0)
		\item Высота матрицы (по умолчанию 0)
		\item Указатель массив указателей на число (по умолчанию $ nullptr $)
	\end{itemize}

	Сама матрица является массивом указателей на числа. 
	Доступ к массиву осуществляется через указатель на него (хранится в стуктуре). 
	Память под числа и массив указателей на них выделяется динамически. 
	% Структура хранения данных указана на рис.\ref{}.
	
	\section*{Описание пользовательских функций}

	\subsection*{GenElement}

	\begin{lstlisting}[label={lst:GenElement}]
		int GenElement(GenType gen_type, int width, int i, int j);
	\end{lstlisting}

	Функция для генерации элементов матрицы.

	Принимает на вход тип генерации, ширину матрицы,
	строку и столбец элемента.

	Возвращает число -- элемент матрицы.

	Алгоритм работы функции:

	В зависимости от типа генерации
	\begin{itemize}
		\item Если GenType::Zero вернуть 0
		\item Если GenType::One вернуть 1
		\item Если GenType::Random вернуть случайное число от 0 до 9
		\item Если GenType::Fill0 вернуть $ i*width+j $
		\item Если GenType::Fill1 вернуть $ i*width+j+1 $
		\item Если GenType::Input вернуть InputElement(i, j)
		\item В остальных случаях вернуть -1
	\end{itemize}

	\subsection*{Input}

	\begin{lstlisting}[label={lst:Input}]
		int Input(const std::string& message = "Input: ", int l = _min_int, int r = _max_int);
	\end{lstlisting}

	Функция для ввода числа.
	Содержит в себе логику по обработке исключений и
	ограничения диапазона допустимых чисел.

	Принимает на вход сообщение, приглашающее к вводу,
	левую и правую границы диапазона ввода.

	Возвращает введённое число

	Алгоритм работы функции:

	\begin{itemize}
		\item Задаём вводимый элемент
		\item Создаём флаг, сигнализирующий конец цикла
		\item Пока флаг поднят:
		\subitem Вывести сообщение
		\subitem Ввести элемент
		\subitem Если ошибка ввода или введённый элемент не входит в допустимый диапазон
		\subsubitem Выводим "Wrong input!"
		\subitem Иначе
		\subsubitem Опускаем флаг цикла
		\subitem Чистим поток cin от мусора
		\item Возвращаем ведённый элемент
	\end{itemize}

	\subsection*{InputElement}

	\begin{lstlisting}[label={lst:InputElement}]
		int InputElement(int i, int j);
	\end{lstlisting}

	Функция для ввода элемента матрицы.
	Является обёрткой над функцией Input().

	Принимает на вход строку и столбец вводимого элемента.

	Возвращает вводимый пользователем элемент

	Алгоритм работы функции:

	\begin{itemize}
		\item Создаём сообщение "Input element [i][j]: ", подставляя вместо i и j входные параметры
		\item Возвращаем Input()
	\end{itemize}

	\subsection*{InputDimensions}

	\begin{lstlisting}[label={lst:InputDimensions}]
		Matrix2 InputDimensions();
	\end{lstlisting}

	Функция для ввода размеров матрицы.
	Является обёрткой над функцией Input().

	Не принимает на вход параметров

	Возвращает матрицу с определёнными размерами.

	Алгоритм работы функции:

	\begin{itemize}
		\item Создаём результирующую матрицу
		\item Записываем в ширину результат ввода
		\item Записываем в высоту результат ввода
		\item Возвращаем результирующую матрицу
	\end{itemize}

	\subsection*{CreateMatrix}

	\begin{lstlisting}[label={lst:CreateMatrix}]
		Matrix2 CreateMatrix(int w, int h, GenType gen_type = GenType::Zero);
	\end{lstlisting}

	Функция для создания матрицы.

	Принимает на вход ширину, высоту и тип генерации матрицы.

	Возвращает матрицу.

	Алгоритм работы функции:

	\begin{itemize}
		\item Создаём результирующую матрицу
		\item Записываем в ширину параметр ширины
		\item Записываем в высоту параметр высоты
		\item Выделяем память под строки (размером $ height $)
		\item Цикл по i от 0 до $ height $
		\begin{itemize}
			\item Выделяем память под элементы строки (размером $ width $)
			\item Цикл по j от 0 до $ width $
			\begin{itemize}
				  \item Генерируем элемент с помощью GenElement
				  \item Записываем элемент в матрицу
			\end{itemize}
		\end{itemize}
		\item Возвращаем результирующую матрицу
	\end{itemize}

	\subsection*{LoadMatrix}

	\begin{lstlisting}[label={lst:LoadMatrix}]
		Matrix2 LoadMatrix(const std::string& file_name);
	\end{lstlisting}

	Функция для загрузки матрицы из файла.

	Принимает на вход имя файла (путь к файлу).

	Возвращает матрицу.

	Алгоритм работы функции:

	\begin{itemize}
		\item Создаём результирующую матрицу
		\item Пытаемся открыть файл
		\item Если файл не открыт, возвращаем пустую матрицу
		\item Считываем из файла параметры ширины и высоты
		\item Выделяем память под строки (размером $ height $)
		\item Цикл по i от 0 до $ height $
		\begin{itemize}
			\item Выделяем память под элементы строки (размером $ width $)
			\item Цикл по j от 0 до $ width $
			\begin{itemize}
				\item Считываем элемент из файла.
				\item Записываем элемент в матрицу
			\end{itemize}
		\end{itemize}
		\item Возвращаем результирующую матрицу
	\end{itemize}

	\subsection*{String}

	\begin{lstlisting}[label={lst:String}]
		std::string String(const Matrix2 &m);
	\end{lstlisting}

	Функция для получения форматированного строкового представления матрицы.

	Принимает на вход константную ссылку на матрицу

	Возвращает форматированное строковое представление матрицы

	Алгоритм работы функции:

	\begin{itemize}
		\item Создаём строковый поток
		\item Цикл по i от 0 до $ height $ матрицы
		\begin{itemize}
			\item Цикл по j от 0 до $ width $ матрицы
			\begin{itemize}
				\item Записываем в поток смещённый вправо на 5 символов элемент матрицы
			\end{itemize}
			\item Записываем в поток символ конца строки
		\end{itemize}
		\item Возвращаем строковое представление потока
	\end{itemize}

	\subsection*{Print}

	\begin{lstlisting}[label={lst:Print}]
		void Print(const Matrix2 &m);
	\end{lstlisting}

	Функция для печати матрицы в стандартный поток вывода.

	Принимает на вход константную ссылку на матрицу

	Не возвращает ничего.

	Алгоритм работы функции:

	\begin{itemize}
		\item Вывести в stdout результат String()
	\end{itemize}

	\subsection*{Sum}

	\begin{lstlisting}[label={lst:Sum}]
		Matrix2 Sum(Matrix2 a, Matrix2 b);
	\end{lstlisting}

	Функция для суммирования двух матриц.

	Принимает на вход две константные ссылки на матрицы.

	Возвращает матрицу -- сумму двух переданных матриц.

	Алгоритм работы функции:

	\begin{itemize}
		\item Если размеры матриц не совпадают, выкидываем исключение.
		\item Создаём результирующую матрицу
		\item Цикл по i от 0 до $ height $
		\begin{itemize}
			\item Цикл по j от 0 до $ width $
			\begin{itemize}
				\item Записываем в результирующую матрицу сумму
				элементов переданных матриц с соответствующими координатами.
			\end{itemize}
		\end{itemize}
		\item Возвращаем результирующую матрицу
	\end{itemize}

	\subsection*{Delete}

	\begin{lstlisting}[label={lst:Delete}]
		void Delete(Matrix2 m);
	\end{lstlisting}

	Функция для удаления матрицы из динамической памяти.

	Принимает на вход ссылку на матрицу.

	Не возвращает ничего.

	Алгоритм работы функции:

	\begin{itemize}
		\item Цикл по i от 0 до $ height $
		\begin{itemize}
			\item Освобождаем память $ i $ -ой строки.
		\end{itemize}
		\item Освобождаем массив указателей
	\end{itemize}

	\subsection*{PrintMainMenu}

	\begin{lstlisting}[label={lst:PrintMainMenu}]
		void PrintMainMenu();
	\end{lstlisting}

	Функция для вывода пользовательского меню.

	Не принимает на вход параметров

	Не возвращает ничего.

	Алгоритм работы функции:

	\begin{itemize}
		\item Вывести меню
	\end{itemize}

	\subsection*{ChooseGenType}

	\begin{lstlisting}[label={lst:ChooseGenType}]
		GenType ChooseGenType();
	\end{lstlisting}

	Функция для пользовательского выбора типа генерации матрицы.

	Не принимает на вход параметров.

	Возвращает выбранный пользователем тип генерации.

	Алгоритм работы функции:

	\begin{itemize}
		\item Выводим меню
		\item Считываем пользовательский ввод (в диапазоне 1--7)
		\item Возвращаем считанное число, приведённое к типу GenType
	\end{itemize}

	\subsection*{TryLoad}

	\begin{lstlisting}[label={lst:TryLoad}]
		bool TryLoad(Matrix2 &m);
	\end{lstlisting}

	Функция для обработки загрузки матрицы.
	Включает в себя обработку исключений.

	Принимает на вход ссылку на матрицу.

	Возвращает bool -- успешность загрузки.

	Алгоритм работы функции:

	\begin{itemize}
		\item Считываем имя файла
		\item Загружаем матрицу
		\item Если размеры матриц не совпадают или матрица пустая
		\begin{itemize}
			\item Выводим сообщение об ошибке
			\item Возвращаем false
		\end{itemize}
		\item Присваиваем переданной матрице загруженную
		\item Возвращаем true
	\end{itemize}

	\subsection*{Menu}

	\begin{lstlisting}[label={lst:Menu}]
		void Menu(Matrix2 &m);
	\end{lstlisting}

	Функция-меню для взаимодействия с пользователем и создания матрицы.

	Принимает на вход ссылку на матрицу.

	Не возвращает ничего.

	Алгоритм работы функции:

	\begin{itemize}
		\item Создаём флаг, сигнализирующий конец цикла
		\item Пока флаг поднят:
		\begin{itemize}
			  \item Считываем тип генерации
			  \item Если тип генерации НЕ файл
			  \subitem Создаём матрицу на основе типа генерации
			  \subitem Опускаем флаг цикла
			  \item Иначе
			  \subitem Флаг цикла = Успешности загрузки матрицы (TryLoad)
		\end{itemize}
		\item Выводим полученную матрицу на экран
	\end{itemize}
	
	\section*{Алгоритм программы}

	\begin{itemize}
		\item Выводим общую инф-ю о программе
		\item Считываем размеры ссумируемых матриц и создаём две матрицы
		с соответствующими размерами
		\item Вызываем меню для первой матрицы
		\item Вызываем меню для второй матрицы
		\item Суммируем матрицы и записываем результат в третью матрицу
		\item Создаём форматированную строку в виде сложения матриц "столбиком"
		\item Выводим форматированную строку в консоль
		\item Выводим форматированную строку в файл output.txt
		\item Удаляем созданные матрицы
	\end{itemize}
	
	\section*{Тестирование программы}
	
	Тесты с описанием действий.
	Обязательно картинки с результатами

	\section*{Выводы}
	
	Вывод
	
	\newpage

	\section*{Приложение А. Листинг программного кода}

	
	\subsection*{main.cpp}
	\lstinputlisting{../main.cpp}
	\newpage

	\subsection*{input.h}
	\lstinputlisting{../input.h}
	\newpage

	\subsection*{input.cpp}
	\lstinputlisting{../input.cpp}
	\newpage

	\subsection*{gen.h}
	\lstinputlisting{../gen.h}
	\newpage

	\subsection*{gen.cpp}
	\lstinputlisting{../gen.cpp}
	\newpage

	\subsection*{matrix.h}
	\lstinputlisting{../matrix.h}
	\newpage

	\subsection*{matrix.cpp}
	\lstinputlisting{../matrix.cpp}
	\newpage

	\subsection*{menu.h}
	\lstinputlisting{../menu.h}
	\newpage

	\subsection*{menu.cpp}
	\lstinputlisting{../menu.cpp}

\end{document}