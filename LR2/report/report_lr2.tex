\documentclass[12pt,a4paper]{article}  % шаблон для статьи, шрифт 12 пт

\usepackage[utf8]{inputenc}  % использование кодировки Юникод UTF-8

\usepackage[compact]{titlesec}  % для titlespacing
% \titlespacing{\заголовок}{слева}{перед}{после}[справа]
\titlespacing*{\section}{0.75cm}{1em}{0.1em}  % отступ заголовка 
\titlespacing*{\subsection}{0.75cm}{1em}{0.1em}

\usepackage{indentfirst}  % отступ первого абзаца
\setlength{\parindent}{0.75cm}

\usepackage[labelsep=endash]{caption}  % тире вместо двоеточия в картинках
\usepackage[russian]{babel}  % пакет поддержки русского языка
\usepackage{graphicx}  % кртинки
\usepackage{color}  % цветной код

\usepackage[T1]{fontenc}
\usepackage{xcolor}  % ???

\usepackage{listings}  % листинги кода из файлов
\lstset{
	language=c++,                   % выбор языка для подсветки (здесь это С++)
	basicstyle=\small,              % размер и начертание шрифта для подсветки кода
	numbers=left,                   % где поставить нумерацию строк (слева\справа)
	numberstyle=\small,             % размер шрифта для номеров строк
	stepnumber=1,                   % размер шага между двумя номерами строк
	numbersep=5pt,                  % как далеко отстоят номера строк от подсвечиваемого кода
	backgroundcolor=\color{white},  % цвет фона подсветки - используем \usepackage{color}
	showspaces=false,               % показывать или нет пробелы специальными отступами
	showstringspaces=false,         % показывать или нет пробелы в строках
	showtabs=false,                 % показывать или нет табуляцию в строках
	frame=false,                    % рисовать рамку вокруг кода
	tabsize=2,                      % размер табуляции по умолчанию равен 2 пробелам
	captionpos=t,                   % позиция заголовка вверху [t] или внизу [b] 
	breaklines=true,                % автоматически переносить строки (да\нет)
	breakatwhitespace=false,        % переносить строки только если есть пробел
	escapeinside={\%*}{*)}          % если нужно добавить комментарии в коде
}

%перенос строк внутри таблиц
\newcommand{\specialcell}[2][c]{%
	\begin{tabular}[#1]{@{}c@{}}#2\end{tabular}}

% нумерация рисунков по номеру главы (e.g. 1.1)
% для отчёта по курсовой
% \renewcommand\thefigure{\arabic{section}.\arabic{figure}}

% Шаблон создания картинки
% \begin{figure}[hpt!]
% 	\centering
% 	\includegraphics[width=0.6\linewidth]{photo/photo1}
% 	\caption{Подпись к рисунку}
% 	\label{имя ссылки на рисунок}
% \end{figure}

\begin{document}

	\thispagestyle{empty}

\begin{center}
    \Large{
    \textbf{МИНОБРНАУКИ РОССИИ}

    \textbf{Санкт-Петербургский государственный}

    \textbf{электротехнический университет «ЛЭТИ»}

    \textbf{им. В.И. Ульянова (Ленина)}

    \textbf{Кафедра САПР}
    }
\end{center}

\topskip=0pt
\vspace*{\fill}

\begin{center}
    \Large{
    \textbf{
    ЛАБОРАТОРНАЯ РАБОТА №N\\
    по дисциплине «Программирование»\\
    Тема: Тема\\
    }
    }
\end{center}

\vspace*{\fill}

\begin{tabular}{lcr}
    Студенты гр. 9892 & \begin{tabular}{p{60mm}} \\ \hline \end{tabular} & Лескин К.А.  \\\\
                      & \begin{tabular}{p{60mm}} \\ \hline \end{tabular} & Миллер В.В.  \\\\
    Преподаватель     & \begin{tabular}{p{60mm}} \\ \hline \end{tabular} & Кузьмин С.А.
    \\\\
\end{tabular}

\begin{center}
    Санкт-Петербург\\
    2020
\end{center}

\newpage
	
	\section*{Цель работы}
	Получение практических навыков разработки алгоритмов для обработки
массивов.

	\section*{Формулировка задания}
	\section*{Формулировка задания}

Разработать программу, позволяющую выполнять различные операции
над базой данных, представленной в виде линейного списка (тема базы
данных и набор операций есть в своём варианте задания).
Курсовая работа «собирается» студентом из функций, объединённых с
помощью меню в головной программе, выполняющей обработку связанного
линейного списка.
В виде отдельных пользовательских функций оформляются части
программы, реализующие операции:
\begin{itemize}
	\item создание списка (выделение памяти, создание и заполнение вводимыми с
	клавиатуры данными элементов списка);
	\item сохранение введённой информации в заданном пользователем файле;
	восстановление списка (заполнение его информацией, считываемой из
	файла, поиск элемента по признаку (признак - одно из полей структуры));
	\item сортировка найденных элементов и вывод информации о них на экран;
	\item корректировка полей записи выбранного элемента (идентификация
	элемента по номеру в выводимом на экран перечне (по номеру указателя
	на элемент));
	\item удаление выбранного элемента (одного из найденных по признаку);
	\item вставка нового элемента (после/перед выбранным).
\end{itemize}
Все задания содержат индивидуальные требования к выполнению.
Курсовая работа оформляется в виде пояснительной записки, в которой
отражены все полученные результаты разработки.

Обеспечить работу компании, осуществляющей грузовые
перевозки на основе наличия:
\begin{itemize}
	\item списка парка грузовиков (марка, грузоподъёмность, максимальная дальность перевозки, плановый пробег в пути за сутки);
	\item списка водителей (ФИО, разрешение на использование марки грузовика);
	\item списка маршрутов перевозки (конечный пункт, дальность, время
	погрузки/разгрузки в конечных пунктах, количество водителей).
	
\end{itemize}
При поступлении очередного заказа (маршрут, дата выезда, масса груза,
пожелание по марке грузовика) необходимо сформировать для поездки
комбинацию грузовик-водитель(-и).

Дополнительно необходимо выдавать информацию:
\begin{itemize}
	\item о свободных водителях на определённую дату;
	\item о свободных грузовиках на определённую дату;
	\item о грузовиках, находящихся на определённом маршруте;
	\item о водителях, находящихся на определённом маршруте;
	\item о плановой дате прибытия грузовика с водителем(-ями).
\end{itemize}

	\section*{Форматы входных и выходных файлов}
	\input{files_formats.tex}

	\section*{Описание структур данных}
	\input{data_structures_description.tex}

	\section*{Описание пользовательских функций}
	\input{functions_description.tex}

	\section*{Алгоритм программы}
	\begin{itemize}
	\item Выводим общую инф-ю о программе
	\item Считываем размеры ссумируемых матриц и создаём две матрицы
	с соответствующими размерами
	\item Вызываем меню для первой матрицы
	\item Вызываем меню для второй матрицы
	\item Суммируем матрицы и записываем результат в третью матрицу
	\item Создаём форматированную строку в виде сложения матриц "столбиком"
	\item Выводим форматированную строку в консоль
	\item Выводим форматированную строку в файл output.txt
	\item Удаляем созданные матрицы
\end{itemize}

	\section*{Тестирование программы}
	Тесты с описанием действий.

Обязательно картинки с результатами.

	\section*{Выводы}
	\input{conclusion.tex}

	\section*{Приложение А. Листинг программного кода}
	Листинг кода

% \subsection*{main.cpp}
% \lstinputlisting{../main.cpp}
% \newpage

\end{document}