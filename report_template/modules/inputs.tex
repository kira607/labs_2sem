\documentclass[12pt,a4paper]{article}  % шаблон для статьи, шрифт 12 пт

\usepackage[utf8]{inputenc}  % использование кодировки Юникод UTF-8

\usepackage[compact]{titlesec}  % для titlespacing
% \titlespacing{\заголовок}{слева}{перед}{после}[справа]
\titlespacing*{\section}{0.75cm}{1em}{0.1em}  % отступ заголовка
\titlespacing*{\subsection}{0.75cm}{1em}{0.1em}

\usepackage{indentfirst}  % отступ первого абзаца
\setlength{\parindent}{0.75cm}

\usepackage[labelsep=endash]{caption}  % тире вместо двоеточия в картинках

\usepackage[russian]{babel}  % пакет поддержки русского языка

\usepackage{graphicx}  % кртинки

\usepackage{color}  % цветной код

\usepackage[T1]{fontenc}
\usepackage{xcolor}  % ???

\usepackage{listings}  % листинги кода из файлов

\lstset{
	language=c++,                   % выбор языка для подсветки (здесь это С++)
	basicstyle=\small,              % размер и начертание шрифта для подсветки кода
	numbers=left,                   % где поставить нумерацию строк (слева\справа)
	numberstyle=\small,             % размер шрифта для номеров строк
	stepnumber=1,                   % размер шага между двумя номерами строк
	numbersep=5pt,                  % как далеко отстоят номера строк от подсвечиваемого кода
	backgroundcolor=\color{white},  % цвет фона подсветки - используем \usepackage{color}
	showspaces=false,               % показывать или нет пробелы специальными отступами
	showstringspaces=false,         % показывать или нет пробелы в строках
	showtabs=false,                 % показывать или нет табуляцию в строках
	frame=false,                    % рисовать рамку вокруг кода
	tabsize=2,                      % размер табуляции по умолчанию равен 2 пробелам
	captionpos=t,                   % позиция заголовка вверху [t] или внизу [b]
	breaklines=true,                % автоматически переносить строки (да\нет)
	breakatwhitespace=false,        % переносить строки только если есть пробел
	escapeinside={\%*}{*)}          % если нужно добавить комментарии в коде
}





