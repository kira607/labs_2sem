\section{Введение}
%\addcontentsline{toc}{section}{Введение}

Целью данной курсовой работы является приобретение навыков разработки и 
отладки многомодульных программ на языке C++. 
Разработка программы, позволяющей 
создать,
обработать, 
вывести 
и удалить 
базу данных на основе линейных списков.

В ходе выполнения работы мы воспользуемся всеми знаниями, 
полученными при выполнении лабораторных работ.
Содержание пояснительной записки к курсовой работе 
соответствует стандартным требованиям к пояснительным 
документам для программного продукта.

В данной курсовой работе предполагается использование функций для
работы с структурой данных "список" \cite{list_defenition}. \\

\textbf{Индивидуальный вариант №10}:

Обеспечить работу компании, осуществляющей грузовые
перевозки на основе наличия:

\begin{itemize}
    \item списка парка грузовиков (марка, грузоподъёмность, максимальная
    дальность перевозки, плановый пробег в пути за сутки);
    \item списка водителей (ФИО, разрешение на использование марки грузовика);
    \item списка маршрутов перевозки (конечный пункт, дальность, время
    погрузки/разгрузки в конечных пунктах, количество водителей).
\end{itemize}

При поступлении очередного заказа (маршрут, дата выезда, масса груза,
пожелание по марке грузовика) необходимо сформировать для поездки
комбинацию грузовик-водитель(-и).

Дополнительно необходимо выдавать информацию:

\begin{itemize}
    \item о свободных водителях на определённую дату;
    \item о свободных грузовиках на определённую дату;
    \item о грузовиках, находящихся на определённом маршруте;
    \item о водителях, находящихся на определённом маршруте;
    \item о плановой дате прибытия грузовика с водителем(-ями).
    
\end{itemize}
