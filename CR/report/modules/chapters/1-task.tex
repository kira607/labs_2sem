\begin{center}
	\Large{
		\textbf{Задание\\ на курсовую работу}
	}
\end{center}

\textbf{Студенты:} Лескин К.А., Миллер В.В.\\

\textbf{Группа:} 9892\\

\textbf{Тема работы:} Разработка электронной картотеки\\

\textbf{Исходные данные:}

Разработать программу, позволяющую выполнять различные операции
над базой данных, представленной в виде линейного списка (тема базы
данных и набор операций есть в своём варианте задания).

Курсовая работа «собирается» студентом из функций, объединённых с
помощью меню в головной программе, выполняющей обработку связанного
линейного списка.

В виде отдельных пользовательских функций оформляются части
программы, реализующие операции:
\begin{itemize}
	\item создание списка (выделение памяти, создание и заполнение вводимыми с
	клавиатуры данными элементов списка);
	\item сохранение введённой информации в заданном пользователем файле;
	\item восстановление списка (заполнение его информацией, считываемой из
	файла, поиск элемента по признаку (признак - одно из полей структуры));
	\item сортировка найденных элементов и вывод информации о них на экран;
	\item корректировка полей записи выбранного элемента (идентификация
	элемента по номеру в выводимом на экран перечне (по номеру указателя
	на элемент));
	\item удаление выбранного элемента (одного из найденных по признаку);
	\item вставка нового элемента (после/перед выбранным).
\end{itemize}

\newpage

\textbf{Индивидуальный вариант №10}:

Обеспечить работу компании, осуществляющей грузовые
перевозки на основе наличия:

\begin{itemize}
	\item списка парка грузовиков (марка, грузоподъёмность, максимальная
	дальность перевозки, плановый пробег в пути за сутки);
	\item списка водителей (ФИО, разрешение на использование марки грузовика);
	\item списка маршрутов перевозки (конечный пункт, дальность, время
	погрузки/разгрузки в конечных пунктах, количество водителей).
\end{itemize}

При поступлении очередного заказа (маршрут, дата выезда, масса груза,
пожелание по марке грузовика) необходимо сформировать для поездки
комбинацию грузовик-водитель(-и).

Дополнительно необходимо выдавать информацию:

\begin{itemize}
	\item о свободных водителях на определённую дату;
	\item о свободных грузовиках на определённую дату;
	\item о грузовиках, находящихся на определённом маршруте;
	\item о водителях, находящихся на определённом маршруте;
	\item о плановой дате прибытия грузовика с водителем(-ями).

\end{itemize}

\newpage

\textbf{Содержание пояснительной записки:}\\
\textbf{Требуемые разделы пояснительной записки:} «Содержание», «Введение», Основные главы,
«Заключение», «Список использованных источников».\\

\textbf{Предполагаемый объем пояснительной записки:}\\
Не менее 30 страниц.\\

\textbf{Дата выдачи задания:} 29.09.2020
\\

\textbf{Дата сдачи курсовой работы:} 23.12.2020
\\

\textbf{Дата защиты курсовой работы:} 29.12.2020
\\

\begin{tabular}{lcr}
	Студенты гр. 9892 & \begin{tabular}{p{60mm}} \\ \hline \end{tabular} & Лескин К.А.  \\\\ 
	& \begin{tabular}{p{60mm}} \\ \hline \end{tabular} & Миллер В.В.  \\\\ 
	Преподаватель     & \begin{tabular}{p{60mm}} \\ \hline \end{tabular} & Кузьмин С.А. \\\\
\end{tabular}