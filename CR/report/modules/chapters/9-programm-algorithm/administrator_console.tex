\subsection{Модуль administrator\_console}

\subsubsection{AdministratorConsole}

\begin{lstlisting}
    AdministratorConsole(DataBases *data_bases);
\end{lstlisting}

\begin{itemize}
    \item \verb|dbs = data_bases;|
\end{itemize}

\subsubsection{Run}

\begin{lstlisting}
    int Run();
\end{lstlisting}

\begin{itemize}
    \item \verb|_loadRealPassword();|
    \item \verb|_inputPassword("Password: ");|
    \item Если \verb|_verifyPassword()| вернул \verb|false|
        \begin{itemize}
            \item \verb|std::cout << "Incorrect password!\n";|
            \item \verb|return 1;|
        \end{itemize}
    \item \verb|return _mainMenu();|
\end{itemize}

\subsubsection{\_loadRealPassword}

\begin{lstlisting}
    void _loadRealPassword();
\end{lstlisting}

\begin{itemize}
    \item \verb|std::ifstream fin("../.pass");|
    \item Если \verb|!fin|
    \begin{itemize}
        \item \verb|std::cerr << ".pass does not exist!\n";|
    \end{itemize}
    \item \verb|fin >> real_password;|
\end{itemize}

\subsubsection{\_uploadPassword}

\begin{lstlisting}
    void _uploadPassword() const;
\end{lstlisting}

\begin{itemize}
    \item \verb|std::ofstream fout("../.pass", std::ios_base::trunc);|
    \item Если \verb|!fout|
    \begin{itemize}
        \item \verb|std::cerr << ".pass does not exist!\n";|
    \end{itemize}
    \item \verb|fout << real_password;|
\end{itemize}

\subsubsection{\_getch}

\begin{lstlisting}
    static int _getch();
\end{lstlisting}

\begin{itemize}
    \item \verb|int ch;|
    \item \verb|struct termios t_old{}, t_new{};|
    \item \verb|tcgetattr(STDIN_FILENO, &t_old);|
    \item \verb|t_new = t_old;|
    \item \verb|t_new.c_lflag &= ~(ICANON ||\verb| ECHO);|
    \item \verb|tcsetattr(STDIN_FILENO, TCSANOW, &t_new);|
    \item \verb|ch = getchar();|
    \item \verb|tcsetattr(STDIN_FILENO, TCSANOW, &t_old);|
    \item \verb|return ch;|
\end{itemize}

\subsubsection{\_inputPassword}

\begin{lstlisting}
    void _inputPassword(const std::string &prompt, bool show_asterisk=true);
\end{lstlisting}

\begin{itemize}
    \item Если \verb|!password.empty()|
        \begin{itemize}
            \item \verb|password.clear();|
        \end{itemize}
    \item \verb|std::cout << prompt;|
    \item \verb|unsigned char ch=0;|
    \item Пока полученный символ не 'Enter' \verb|((ch = _getch()) != RETURN)|
        \begin{itemize}
            \item Если \verb|ch == BACKSPACE|
            \begin{itemize}
                \item Если \verb|password.length() != 0|
                \item \ \ Если\verb|show_asterisk|
                \item \ \ \ \ \verb|std::cout << "\b \b";|
                \item \verb|password.resize(password.length() - 1);|
            \end{itemize}
            \item Иначе
            \begin{itemize}
                \item \verb|password += ch;|
                \item Если\verb|show_asterisk|
                \item \ \ \verb|std::cout << "\b \b";|
            \end{itemize}
        \end{itemize}
    \item \verb|std::cout << "\n";|
\end{itemize}

\subsubsection{\_verifyPassword}

\begin{lstlisting}
    bool _verifyPassword() const;
\end{lstlisting}

\begin{itemize}
    \item \verb|std::string hashed_password = sha256(password);|
    \item \verb|return hashed_password == real_password;|
\end{itemize}

\subsubsection{\_mainMenu}

\begin{lstlisting}
    int _mainMenu();
\end{lstlisting}

\begin{itemize}
    \item Пока \verb|true|
    \begin{itemize}
        \item \verb|std::cout << "\nChoose Data Base" << "\n";|
        \item \verb|std::cout << "1 Choose Truck Data Base" << "\n";|
        \item \verb|std::cout << "2 Choose Driver Data Base" << "\n";|
        \item \verb|std::cout << "3 Choose Route Data Base" << "\n";|
        \item \verb|std::cout << "4 Change password\n";|
        \item \verb|std::cout << "0 Exit" << "\n";|
        \item \verb|int option = InputInt("Input: ");|
        \item Смотрим на \verb|option|
        \begin{itemize}
            \item Если 1: \verb|_truckMenu(); break;|
            \item Если 2: \verb|_driverMenu(); break;|
            \item Если 3: \verb|_routeMenu(); break;|
            \item Если 4: \verb|_changePassword(); break;|
            \item Если 0: \verb|dbs->CloseAll(); return 0;|
            \item По умолчанию: \verb|std::cout << "\nIncorrect input\n\n";|
        \end{itemize}
    \end{itemize}
\end{itemize}

\subsubsection{\_truckMenu}

\begin{lstlisting}
    void _truckMenu() const;
\end{lstlisting}

\begin{itemize}
    \item Пока \verb|true|
    \begin{itemize}
        \item \verb|std::cout << "\nTruck Data Base" << "\n";|
        \item \verb|std::cout << "1 Print all" << "\n";|
        \item \verb|std::cout << "2 Edit" << "\n";|
        \item \verb|std::cout << "3 Add" << "\n";|
        \item \verb|std::cout << "4 Delete" << "\n";|
        \item \verb|std::cout << "0 Exit" << "\n";|
        \item \verb|int option = InputInt("Input: ");|
        \item Смотрим на \verb|option|
        \begin{itemize}
            \item Если 1: \verb|dbs->truck_db.PrintAll(); break;|
            \item Если 2: 
            \begin{itemize}
                \item \verb|int index = InputInt("Choose item by index: ", 0, dbs->truck_db.list.size-1);|
                \item \verb|dbs->truck_db.Edit(index);|
                \item \verb|break;|
            \end{itemize}
            \item Если 3: \verb|dbs->truck_db.Add(); break;|
            \item Если 4: \verb|dbs->truck_db.Delete(InputInt("Index: ")); break;|
            \item Если 0: \verb|return;|
            \item По умолчанию: \verb|std::cout << "\nIncorrect input\n\n";|
        \end{itemize}
    \end{itemize}
\end{itemize}

\subsubsection{\_driverMenu}

\begin{lstlisting}
    void _driverMenu() const;
\end{lstlisting}

\begin{itemize}
    \item Пока \verb|true|
    \begin{itemize}
        \item \verb|std::cout << "\nDriver Data Base" << "\n";|
        \item \verb|std::cout << "1 Print all" << "\n";|
        \item \verb|std::cout << "2 Edit" << "\n";|
        \item \verb|std::cout << "3 Add" << "\n";|
        \item \verb|std::cout << "4 Delete" << "\n";|
        \item \verb|std::cout << "0 Exit" << "\n";|
        \item \verb|int option = InputInt("Input: ");|
        \item Смотрим на \verb|option|
        \begin{itemize}
            \item Если 1: \verb|dbs->driver_db.PrintAll(); break;|
            \item Если 2: 
            \begin{itemize}
                \item \verb|int index = InputInt("Choose item by index: ", 0, dbs->driver_db.list.size-1);|
                \item \verb|dbs->driver_db.Edit(index);|
                \item \verb|break;|
            \end{itemize}
            \item Если 3: \verb|dbs->driver_db.Add(); break;|
            \item Если 4: \verb|dbs->driver_db.Delete(InputInt("Index: ")); break;|
            \item Если 0: \verb|return;|
            \item По умолчанию: \verb|std::cout << "\nIncorrect input\n\n";|
        \end{itemize}
    \end{itemize}
\end{itemize}

\subsubsection{\_routeMenu}

\begin{lstlisting}
    void _routeMenu() const;
\end{lstlisting}

\begin{itemize}
    \item Пока \verb|true|
    \begin{itemize}
        \item \verb|std::cout << "\nRoute Data Base" << "\n";|
        \item \verb|std::cout << "1 Print all" << "\n";|
        \item \verb|std::cout << "2 Edit" << "\n";|
        \item \verb|std::cout << "3 Add" << "\n";|
        \item \verb|std::cout << "4 Delete" << "\n";|
        \item \verb|std::cout << "0 Exit" << "\n";|
        \item \verb|int option = InputInt("Input: ");|
        \item Смотрим на \verb|option|
        \begin{itemize}
            \item Если 1: \verb|dbs->route_db.PrintAll(); break;|
            \item Если 2: 
            \begin{itemize}
                \item \verb|int index = InputInt("Choose item by index: ", 0, dbs->route_db.list.size-1);|
                \item \verb|dbs->route_db.Edit(index);|
                \item \verb|break;|
            \end{itemize}
            \item Если 3: \verb|dbs->route_db.Add(); break;|
            \item Если 4: \verb|dbs->route_db.Delete(InputInt("Index: ")); break;|
            \item Если 0: \verb|return;|
            \item По умолчанию: \verb|std::cout << "\nIncorrect input\n\n";|
        \end{itemize}
    \end{itemize}
\end{itemize}

\subsubsection{\_changePassword}

\begin{lstlisting}
    void _changePassword();
\end{lstlisting}

\begin{itemize}
    \item \verb|_inputPassword("Old password: ");|
    \item Если \verb|!_verifyPassword()|
    \begin{itemize}
        \item \verb|std::cerr << "Invalid password!\n";|
        \item \verb|return;|
    \end{itemize}
    \item \verb|_inputPassword("New Password: ");|
    \item \verb|std::string hashed_password = sha256(password);|
    \item \verb|real_password = hashed_password;|
    \item \verb|_uploadPassword();|
    \item \verb|std::cout << "Password changed\n";|
\end{itemize}

