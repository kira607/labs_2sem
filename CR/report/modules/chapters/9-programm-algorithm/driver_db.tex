\subsection{Модуль driver\_db}

\subsubsection{DriverDataBase}

\begin{lstlisting}
    DriverDataBase(ScheduleDataBase *schedule_p, const std::string &db_path_ = "../dbs/driverdb.csv");
\end{lstlisting}

\begin{itemize}
    \item \verb|schedule = schedule_p;|
    \item \verb|db_path = db_path_;|
    \item \verb|_loadDataBase();|
\end{itemize}

\subsubsection{PrintAll}

\begin{lstlisting}
    void PrintAll() const;
\end{lstlisting}

\begin{itemize}
    \item \verb|std::cout << std::left|\\
    \verb|<< std::setw(5)  << "id"|\\
    \verb|<< std::setw(11) << "surname"|\\
    \verb|<< std::setw(11) << "name"|\\
    \verb|<< std::setw(16) << "patronymic"|\\
    \verb|<< std::setw(12) << "truck_brand\n";|
    \item Цикл по \verb|i| от \verb|0| до \verb|list.size| 
        \begin{itemize}
            \item \verb|Print(i);|
        \end{itemize}
\end{itemize}

\subsubsection{Print}

\begin{lstlisting}
    void Print(int index) const;
\end{lstlisting}

\begin{itemize}
    \item \verb|list._check_index(index);|
    \item \verb|auto p = list.Get(index);|
    \item \verb|std::cout << std::left|\\
          \verb|<< std::setw(4)  << p->id << " "|\\
          \verb|<< std::setw(10) << p->surname << " "|\\
          \verb|<< std::setw(10) << p->name << " "|\\
          \verb|<< std::setw(15) << p->patronymic << " "|\\
          \verb|<< std::setw(13) << str(p->truck_brand) << "\n";|
\end{itemize}

\subsubsection{Find}

\begin{lstlisting}
    int *Find(Request *request) const;
\end{lstlisting}

\begin{itemize}
    \item \verb|Driver *pDriver = list.head;|
    \item \verb|int *drivers_ids = new int[request->target_route->drivers];|
    \item \verb|int found_free_drivers = 0;|
    \item Пока \verb|pDriver|
        \begin{itemize}
            \item Если \verb|(request->truck_brand != pDriver->truck_brand) |||\\
            \verb|(!schedule->IsFree(pDriver, request))|
                \begin{itemize}
                    \item \verb|pDriver = pDriver->next;|
                    \item \verb|continue;|
                \end{itemize}
            \item \verb|drivers_ids[found_free_drivers] = pDriver->id;|
            \item \verb|++found_free_drivers;|
            \item Если \verb|found_free_drivers < request->target_route->drivers|
                \begin{itemize}
                    \item \verb|pDriver = pDriver->next;|
                    \item \verb|continue;|
                \end{itemize}
            \item \verb|return drivers_ids;|
        \end{itemize}
    \item \verb|return nullptr;|
\end{itemize}

\subsubsection{Edit}

\begin{lstlisting}
    void Edit(int index);
\end{lstlisting}

\begin{itemize}
    \item \verb|list._check_index(index);|
    \item \verb|Driver *target_element = list.Get(index);|
    \item Пока \verb|true|
        \begin{itemize}
            \item \verb|std::cout << "------------------------\n";|
            \item \verb|Print(index);|
            \item \verb|std::cout << "------------------------\n";|
            \item \verb|std::cout << "Choose filed:\n";|
            \item \verb|std::cout << "1 surname" << "\n";|
            \item \verb|std::cout << "2 name" << "\n";|
            \item \verb|std::cout << "3 patronymic" << "\n";|
            \item \verb|std::cout << "4 truck brand" << "\n";|
            \item \verb|std::cout << "0 Finish edit" << "\n";|
            \item \verb|int option = InputInt("Input: ");|
            \item Смотрим на \verb|option|
            \begin{itemize}
                \item Если \verb|1:|
                \item \verb|    target_element->surname = InputStr("New value: ");|
                \item \verb|    break;|
                \item Если \verb|2:|
                \item \verb|    target_element->name = InputStr("New value: ");|
                \item \verb|    break;|
                \item Если \verb|3:|
                \item \verb|    target_element->patronymic = InputStr("New value: ");|
                \item \verb|    break;|
                \item Если \verb|4:|
                \item \verb|    target_element->truck_brand = InputTB("New value: ");|
                \item \verb|    break;|
                \item Если \verb|0: _updateDbFile(); return;|
                \item По умолчанию \verb|: std::cout << "\nIncorrect input\n\n";|
            \end{itemize}
        \end{itemize}
\end{itemize}

\subsubsection{Add}

\begin{lstlisting}
    void Add();
\end{lstlisting}

\begin{itemize}
    \item \verb|Driver new_element;|
    \item \verb|new_element.id = list.Get(list.size-1)->id + 1;|
    \item \verb|new_element.surname = InputStr("surname: ");|
    \item \verb|new_element.name = InputStr("name: ");|
    \item \verb|new_element.patronymic = InputStr("patronymic: ");|
    \item \verb|new_element.truck_brand = InputTB("truck brand: ");|
    \item \verb|list.Add(new_element);|
    \item \verb|_updateDbFile();|
\end{itemize}

\subsubsection{Delete}

\begin{lstlisting}
    void Delete(int index);
\end{lstlisting}

\begin{itemize}
    \item \verb|list.Delete(index);|
    \item Цикл по \verb|i| от \verb|index| до \verb|list.size| 
    \begin{itemize}
        \item \verb|list.Get(i)->id = i;|
    \end{itemize}
    \item \verb|_updateDbFile();|
\end{itemize}

\subsubsection{Exit}

\begin{lstlisting}
    void Exit();
\end{lstlisting}

\begin{itemize}
    \item \verb|list.Free();|
\end{itemize}

\subsubsection{\_loadDataBase}

\begin{lstlisting}
    void _loadDataBase();
\end{lstlisting}

\begin{itemize}
    \item \verb|Driver driver{};|
    \item \verb|io::CSVReader<5> in(db_path);|
    \item \verb|in.read_header(io::ignore_extra_column, "id", "name", "surname", "patronymic", "brand_code");|
    \item \verb|int brand_code;|
    \item Пока \verb|in.read_row(driver.id, driver.name, driver.surname, driver.patronymic, brand_code)|
        \begin{itemize}
            \item \verb|driver.truck_brand = static_cast<TruckBrand>(brand_code);|
            \item \verb|list.Add(driver);|
        \end{itemize}
\end{itemize}

\subsubsection{\_updateDbFile}

\begin{lstlisting}
    void _updateDbFile() const;
\end{lstlisting}

\begin{itemize}
    \item \verb|std::ofstream fout(db_path, std::ios_base::trunc);|
    \item \verb|fout << "id,name,surname,patronymic,brand_code\n";|
    \item \verb|Driver *p = list.head;|
    \item Пока \verb|p|
        \begin{itemize}
            \item 
            \verb|fout << p->id << ","|\\
            \verb|     << p->name << ","|\\
            \verb|     << p->surname << ","|\\
            \verb|     << p->patronymic << ","|\\
            \verb|     << static_cast<int>(p->truck_brand);|
            \item \verb|fout << "\n";|
            \item \verb|p = p->next;|
        \end{itemize}
\end{itemize}

