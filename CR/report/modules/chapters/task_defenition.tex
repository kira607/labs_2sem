\section*{Формулировка задания}

Разработать программу, позволяющую выполнять различные операции
над базой данных, представленной в виде линейного списка (тема базы
данных и набор операций есть в своём варианте задания).
Курсовая работа «собирается» студентом из функций, объединённых с
помощью меню в головной программе, выполняющей обработку связанного
линейного списка.
В виде отдельных пользовательских функций оформляются части
программы, реализующие операции:
\begin{itemize}
	\item создание списка (выделение памяти, создание и заполнение вводимыми с
	клавиатуры данными элементов списка);
	\item сохранение введённой информации в заданном пользователем файле;
	восстановление списка (заполнение его информацией, считываемой из
	файла, поиск элемента по признаку (признак - одно из полей структуры));
	\item сортировка найденных элементов и вывод информации о них на экран;
	\item корректировка полей записи выбранного элемента (идентификация
	элемента по номеру в выводимом на экран перечне (по номеру указателя
	на элемент));
	\item удаление выбранного элемента (одного из найденных по признаку);
	\item вставка нового элемента (после/перед выбранным).
\end{itemize}
Все задания содержат индивидуальные требования к выполнению.
Курсовая работа оформляется в виде пояснительной записки, в которой
отражены все полученные результаты разработки.

Обеспечить работу компании, осуществляющей грузовые
перевозки на основе наличия:
\begin{itemize}
	\item списка парка грузовиков (марка, грузоподъёмность, максимальная дальность перевозки, плановый пробег в пути за сутки);
	\item списка водителей (ФИО, разрешение на использование марки грузовика);
	\item списка маршрутов перевозки (конечный пункт, дальность, время
	погрузки/разгрузки в конечных пунктах, количество водителей).
	
\end{itemize}
При поступлении очередного заказа (маршрут, дата выезда, масса груза,
пожелание по марке грузовика) необходимо сформировать для поездки
комбинацию грузовик-водитель(-и).

Дополнительно необходимо выдавать информацию:
\begin{itemize}
	\item о свободных водителях на определённую дату;
	\item о свободных грузовиках на определённую дату;
	\item о грузовиках, находящихся на определённом маршруте;
	\item о водителях, находящихся на определённом маршруте;
	\item о плановой дате прибытия грузовика с водителем(-ями).
\end{itemize}